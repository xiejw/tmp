% {{{1 Basic Configuration

% {{{3 Load Fonts
\font\ninett=cmtt9

% {{{1 Use opamc for macros. See: https://tug.org/TUGboat/tb34-1/tb106olsak-opmac.pdf

% {{{3 Import OPmac
\input opmac

% {{{3 Defines Colors.
% See http://www.december.com/html/spec/cmykshades.html
\def\StarbuckGreen{\setcmykcolor{1.0 0 0.5 0.6}}

% {{{3 Margin Notes.
\fixmnotes\right \mnotesize=1.7in \def\mnotehook{\typosize[8/10]\it}
\mnoteindent=20pt
\margins/1 letter (1,2.5,,)in

% {{{3 Hyper Links.
\hyperlinks \Green \Blue % Internal green, external blue.

% {{{3 Ajudst verbatim.
\def\tthook{\typosize[9/11]}  % Adjust size for verbatim.

% {{{1 Customized Macors
%
% {{{3 Defines a section-like date.
%
% - Insert a title in bold.
% - Skip 5pt vertically.
% - The first paragraph has no indent. This is done via the \everypar, which is
%   invoked at the start of each paragraph. It sets an empty box at first and
%   resets itself.
\def\date#1{\vskip10pt \noindent{\bf #1}\hfill\vskip 5pt
  \everypar={{\setbox0\lastbox}\everypar={}}}

% {{{3 Defines a \symbol macro for coding related nouns
%
% * It starts with a group and redfines the undercore (_) to be normal char
% * As all arguments are read when process macro. The macro should be defined as
%   two macros, so the underscore in the argument can be handled correctly.
% * \relax makes the \catcode effective immediately. A {} (empty group) or empty
%   space can work also. This is the space-after-number rule (See TexBook Page
%   208).
% * In the second macro, we use the OPmac local color to change the group as
%   green color.
\def\symbol{\begingroup\catcode`\_=12\relax\symbolimpl}
\def\symbolimpl#1{{\ninett #1}\endgroup}
